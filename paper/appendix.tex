\documentclass[english, twocolumn, 10pt, aps, superscriptaddress, floatfix, prb, citeautoscript]{revtex4-1}
\pdfoutput=1
\usepackage{xr}
\externaldocument{zigzag}
\usepackage[utf8]{inputenc}
\usepackage[T1]{fontenc}
\usepackage{verbatim}
\usepackage{units}
\usepackage{mathtools}
\usepackage{amsmath}
\usepackage{amssymb}
\usepackage{graphicx}
\usepackage{wasysym}
\usepackage{layouts}
\usepackage{siunitx}
\usepackage{bm}
\usepackage{xcolor}
\usepackage[colorlinks, citecolor={blue!50!black}, urlcolor={blue!50!black}, linkcolor={red!50!black}]{hyperref}
\usepackage{bookmark}
\usepackage{tabularx}
\usepackage{microtype}
\usepackage{babel}
\hypersetup{pdfauthor={Quantum Tinkerer},pdftitle={Enhanced proximity effect in zigzag-shaped Majorana Josephson junctions.}}

\setcounter{secnumdepth}{4}
\setcounter{tocdepth}{4}

\DeclareMathOperator{\e}{e}
\DeclareMathOperator{\de}{d\!}
\DeclareMathOperator{\Tr}{Tr}
\DeclareMathOperator{\diag}{diag}
\DeclareMathOperator{\Res}{Res}
\DeclareMathOperator{\sgn}{sgn}
\DeclareMathOperator{\Pf}{Pf}
\DeclareMathOperator{\Det}{Det}
\DeclareMathOperator{\rank}{rank}
\DeclareMathOperator{\im}{Im}
\DeclareMathOperator{\re}{Re}

\renewcommand{\comment}[2]{#2}
% Uncomment the following line for paragraph descriptions to appear in the file.
% \renewcommand{\comment}{\paragraph}

\DeclarePairedDelimiter\abs{\lvert}{\rvert}
\DeclarePairedDelimiter\norm{\lVert}{\rVert}

\makeatletter
\let\oldabs\abs
\def\abs{\@ifstar{\oldabs}{\oldabs*}}
\let\oldnorm\norm
\def\norm{\@ifstar{\oldnorm}{\oldnorm*}}
\makeatother

\newcommand{\ev}[1]{\langle#1\rangle}
\newcommand{\bra}[1]{\langle#1|}
\newcommand{\ket}[1]{|#1\rangle}
\newcommand{\bracket}[2]{\langle#1|#2\rangle}

\newcolumntype{L}[1]{>{\raggedright\arraybackslash}p{#1}}
\newcolumntype{C}[1]{>{\centering\arraybackslash}p{#1}}
\newcolumntype{R}[1]{>{\raggedleft\arraybackslash}p{#1}}


\begin{document}


\appendix

\section{Zigzag device with a single superconductor}\label{appendix:NS_junction}

\comment{We run the simulation and observe that the phase diagrams are the same but the zigzag device is still better.}
A zigzag-shaped device with a single superconductor shows a similar enhancement of the superconducting gap as the Josephson junction devices.
We demonstrate this by simulating a straight and a zigzag device and computing $E_\textrm{gap}(B_x, \; \mu)$ in Fig.~\ref{fig:ns_junction}.
We observe that the order of magnitude increase of $E_\textrm{gap}$ also occurs upon the introduction of a zigzag geometry.

\begin{figure}[!htb]
\centering
\includegraphics[width=0.9\columnwidth]{figures/phasediagrams_NS.pdf}
\caption{Same as Fig.~\ref{fig:phasediagrams} (c) and (d) but with one superconductor instead of two.
We observe that the zigzag geometry also increases the gap in this case.
\label{fig:ns_junction}}
\end{figure}

\section{Direction of the magnetic field}\label{appendix:direction_B}

\comment{Zigzag geometry removes the need for precise field alignment.}
Because the zigzag geometry cuts off long trajectories---leaving only short trajectories---this device requires a less precise alignment of the magnetic field.
To show this, we perform a simulation of a straight and zigzag device with a rotated field (see Fig.~\ref{fig:tilted_angle_phase_diagram}).
We observe that the zigzag device still has a sizeable gap with a $10^\circ$ misaligned magnetic field, wherea a $1^\circ$ misalignment makes a straight device gapless.

\comment{Likewise a inhomogeneous magnetic field does not matter.}
Strong screening of the magnetic field by the superconductors may distort the magnetic field pattern.
To determine whether this effect degrades the device quality, we simulate a device with the magnetic field parallel to the NS interface, shown in Fig.~\ref{fig:syst_snaking_magnetic_field}.
We find that the resulting gap is comparable to that of a device with the magnetic field purely in the $x$-direction, as shown in Fig.~\ref{fig:snaking_magnetic_field}.

\begin{figure}[!htb]
\centering
\includegraphics[width=0.9\columnwidth]{figures/tilted_angle_phase_diagram.pdf}
\caption{The topological gap as a function of the magnetic field magnitude $B$ and its angle $\theta=\arctan{B_y/B_x}$ for a straight device (a) and a zigzag device (b).
Note that the colormaps and the $\theta$-axes differ between panels: the zigzag-shaped device becomes gapless at $\theta$ an order of magnitude larger compared to the straight device.
\label{fig:tilted_angle_phase_diagram}}
\end{figure}

\begin{figure}[!htb]
\centering
\includegraphics{figures/syst_snaking_magnetic_field.pdf}
\caption{
The magnetic field pattern emulating strong screening by the superconductors.
\label{fig:syst_snaking_magnetic_field}}
\end{figure}

\begin{figure}[!htb]
\centering
\includegraphics{figures/snaking_magnetic_field.pdf}
\caption{
Comparison of the zigzag device's band structures with the magnetic field following the shape of the junction (orange) (see Fig.~\ref{fig:syst_snaking_magnetic_field}) and a magnetic field in the $x$-direction (blue) for different values of $z_y$.
\label{fig:snaking_magnetic_field}}
\end{figure}

\section{Interface transparency vs.~path length cut-off}

The zigzag geometry both increases the interface transparency (because particles typically hit the superconductor
closer to normal incidence) as well as sets an upper limit to the trajectory length.
The change in the interface transparency alone is not sufficient to explain the increase of the topological gap.
To verify this, we compute the dispersion of a system with $z_x = z_y = \SI{80}{\nm}$ comparable to the Fermi wavelength $\lambda_\textrm{F}$ and $W=\SI{400}{\nm}$, so that the angle of the boundaries is sufficiently large, but the boundary modulation is much smaller than the width.
We observe that $E_\textrm{gap}$ becomes $\SI{6.1}{\mu \eV}$, a two-fold increase compared to the straight system, not the order of magnitude increase resulting from the cutting off of long trajectories.

\end{document}