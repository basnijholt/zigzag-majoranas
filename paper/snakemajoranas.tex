\documentclass[english, twocolumn, 10pt, aps, superscriptaddress, floatfix, prb, citeautoscript]{revtex4-1}
\pdfoutput=1
\usepackage[utf8]{inputenc}
\usepackage[T1]{fontenc}
\usepackage{verbatim}
\usepackage{units}
\usepackage{mathtools}
\usepackage{amsmath}
\usepackage{amssymb}
\usepackage{graphicx}
\usepackage{wasysym}
\usepackage{layouts}
\usepackage{siunitx}
\usepackage{bm}
\usepackage{xcolor}
\usepackage[colorlinks, citecolor={blue!50!black}, urlcolor={blue!50!black}, linkcolor={red!50!black}]{hyperref}
\usepackage{bookmark}
\usepackage{tabularx}
\usepackage{microtype}
\usepackage{babel}
\hypersetup{pdfauthor={Quantum Tinkerer},pdftitle={Increasing the topological gap by an order of magnitude through the geomerty for Majorana SNS junctions.}}

\setcounter{secnumdepth}{4}
\setcounter{tocdepth}{4}

\DeclareMathOperator{\e}{e}
\DeclareMathOperator{\de}{d\!}
\DeclareMathOperator{\Tr}{Tr}
\DeclareMathOperator{\diag}{diag}
\DeclareMathOperator{\Res}{Res}
\DeclareMathOperator{\sgn}{sgn}
\DeclareMathOperator{\Pf}{Pf}
\DeclareMathOperator{\Det}{Det}
\DeclareMathOperator{\rank}{rank}
\DeclareMathOperator{\im}{Im}
\DeclareMathOperator{\re}{Re}
\newcommand{\kx}{k_x}
\newcommand{\ky}{k_y}
\newcommand{\meff}{m_\text{eff}}

\renewcommand{\comment}[2]{#2}
% Uncomment the following line for paragraph descriptions to appear in the file.
\renewcommand{\comment}{\paragraph}

\DeclarePairedDelimiter\abs{\lvert}{\rvert}
\DeclarePairedDelimiter\norm{\lVert}{\rVert}

\makeatletter
\let\oldabs\abs
\def\abs{\@ifstar{\oldabs}{\oldabs*}}
\let\oldnorm\norm
\def\norm{\@ifstar{\oldnorm}{\oldnorm*}}
\makeatother

\newcommand{\ev}[1]{\langle#1\rangle}
\newcommand{\bra}[1]{\langle#1|}
\newcommand{\ket}[1]{|#1\rangle}
\newcommand{\bracket}[2]{\langle#1|#2\rangle}

\newcolumntype{L}[1]{>{\raggedright\arraybackslash}p{#1}}
\newcolumntype{C}[1]{>{\centering\arraybackslash}p{#1}}
\newcolumntype{R}[1]{>{\raggedleft\arraybackslash}p{#1}}


\begin{document}


\title{Increasing the topological gap by an order of magnitude through the geomerty for Majorana SNS junctions.}

\author{Quantum Tinkerer}
\affiliation{Kavli Institute of Nanoscience, Delft University of Technology, P.O. Box 4056, 2600 GA Delft, The Netherlands}
\email[Electronic address: ]{quantumtinkerer@tudelft.nl}

\date{\today}

\begin{abstract}
We are solving a problem of soft gap in high density SNS Majorana junctions \cite{pientka2017topological}.
We do this by introducing a zigzag geometry.
This works because the maximum trajactory length is cut-off due to the geomerty and therefore the minimal energy gap is given by a simple formula, which we also verify using numerics.
In addition to having a large minimum energy gap, the abscence of long length trajectories means that the localization length is very short as well.
\end{abstract}

\maketitle

%%██████████████████████████████████████████████████████████████████████████
%%██ Introduction
%%██████████████████████████████████████████████████████████████████████████
\section{Introduction}
In order to create a stable platform for quantum computing, Majorana bound states are one of the contenders currently being pursued. Currently, most efforts to create pairs of Majorana's are based on hybrid NS structures, where a lengthy strip of semiconductor is interfaced with one, or two superconductor structures. Assuming the right parameters, the semiconductor becomes topological and Majorana bound states appear at the ends of the segment. In practice however, multiple problems arise making even the observation, let alone the manipulation of Majorana's difficult.

We claim that one of the problems lying in the way of a stable system, is the appearance of a soft gap; the gap is greatly reduced for states with momentum directed along the length of the strip. We claim that this is due to, from a semiclassical perspective, long paths through the semiconductor without interruption by the superconductor. We show that introducing a zigzag or snake like geometry for the semiconductor removes these long trajectories and avoids the appearance of a soft gap.

%%██████████████████████████████████████████████████████████████████████████
%%██ Physical picture
%%██████████████████████████████████████████████████████████████████████████
\section{Physical picture}
We consider a system [\ref{fig:setup}] consisting of a two dimensional strip of semiconductor, with superconductors adjacent on both sides.
A magnetic field pointed along the x direction is applied.
We model the system with a BdG hamiltonian; the normal part also has Rashba spin-orbit coupling and a Zeeman field, and the superconductors differ in superconducting phase.
For the zigzag geometry, we take a sawtooth like pattern wher $z_x$ is periodicity, $z_y$ the amplitude and W the width of the junction.
\begin{small}
\begin{align}
    H = \left(\frac{\hbar^2\left(\kx^2 + \ky^2\right)}{2\meff} - \mu\right)\tau_z+
        E_\text{z} \sigma_z+
        \alpha \left( \ky \sigma_x - \kx \sigma_y \right) \tau_z +
        \Delta \tau_y
\end{align}
\label{eq:hamiltonian}
\end{small}

\begin{figure}[!htb]
\includegraphics[width=0.95\columnwidth]{figures/systemPicture.pdf}
\caption{Setups. Figure of a straight and zigzag system, including trajectories.
\label{fig:setup}}
\end{figure}

%%██████████████████████████████████████████████████████████████████████████
%%██ Analytical estimate
%%██████████████████████████████████████████████████████████████████████████
\section{Analytical estimate}
We claim that the hardening of the gap is due to the cutoff of long trajectories. 
In order to verify this is the case, we take a ribbon SNS junction with magnetic field rotated such that it corresponds to a rising section of the zigzag.
We then compute the scattering matrix to find the Andreev bound spectrum as a function of the momentum along the strip direction.
We cut off the spectrum at the momentum corresponding to the proposed cutoff in the zigzag, and compare it to the numerically obtained gap for a true zigzag system.


\begin{align}
    S &= \left(
    \begin{array}{rr}
    r_{ll}&t_{rl}\\
    t_{lr}&r_{rr}\\
    \end{array}
    \right) =
    \left(
    \begin{array}{rr}
    \beta_+ & e^{-i q W} \beta_-\\
    e^{-i q W} \beta_- & e^{-2 i q W} \beta_+\\
    \end{array}
    \right) 
    \\
    \beta_\pm &= \left(
    \begin{array}{rr}
    e^{i \nu_{\arg}}\left(\omega^\pm_1 - \omega^\pm_2\right) & (\omega^\pm _1 + \omega^\pm _2)\\
    -(\omega^\pm _1 + \omega^\pm _2) & e^{-i \nu _{\arg }} \left(\omega^\pm _2 - \omega^\pm _1\right)\\
    \end{array}
    \right)\\
    \omega^\pm_j &= \frac{1}{4} \left(\gamma _{j} \pm \delta _{j}\right)
\end{align}

\begin{align}
    \gamma_j &= \frac{q+i k_{j} \tan \left(\frac{W k_{j}}{2}\right)}{q-i k_{j} \tan \left(\frac{W k_{j}}{2}\right)} \\
    \delta_j &= \frac{q-i k_{j} \cot \left(\frac{W k_{j}}{2}\right)}{q+i k_{j} \cot \left(\frac{W k_{j}}{2}\right)}\\
    e^{i \nu_{\arg}} &= \frac{E_\text{z} e^{i \theta }-i \alpha  k_x}{\sqrt{E_\text{z}^2+\alpha  k_x \left(\alpha  k_x-2 E_\text{z} \sin (\theta )\right)}}
\end{align}
\begin{footnotesize}
\begin{align}
    q &= \left[ \frac{2 m_\ast}{\hbar ^2}\mu_s - k_x^2 \right]^\frac{1}{2}\\
    k_1 &= \left[ \frac{2 m_\ast}{\hbar^2} \left(\mu_n-\sqrt{E_z^2-2 \alpha  E_z \sin (\theta ) k_x+\alpha ^2 k_x^2}\right) - k_x^2 \right]^\frac{1}{2}\\
    k_2 &= \left[ \frac{2 m_\ast}{\hbar^2} \left(\mu_n+\sqrt{E_z^2-2 \alpha  E_z \sin (\theta ) k_x+\alpha ^2 k_x^2}\right) - k_x^2 \right]^\frac{1}{2}
\end{align}
\end{footnotesize}



%%██████████████████████████████████████████████████████████████████████████
%%██ Numerical bandstructures
%%██████████████████████████████████████████████████████████████████████████
\section{Numerical bandstructures}

\comment{We consider a 2D SNS junction with a Rashba-BdG Hamiltonian.}

\comment{We discretize the model and simulate it with Kwant.}


%%██████████████████████████████████████████████████████████████████████████
%%██ Calculating the bandstuctures
%%██████████████████████████████████████████████████████████████████████████
\section{Calculating the bandstuctures.}
In figure \ref{fig:bandstuctures}, We calculate the bandstructure for a zigzag system with varying amplitude using Kwant.
As the zigzag introduces super cell, we see a folded bandstructure.
For comparisons sake, we take the same cell size for the ribbon structure so the bandstructure is folded in the same way.
The introduction of a zigzag has a striking effect: the bands flatten out, and more importantly, the gap size is increased tenfolds.
We claim this effect is due to the fact that the modes where the gap is smallest, when the momentum is almost completely focused along the strip ($k_x=k_F$), are cut off by the zigzag geometry.
Additionally in the presence of normal reflection, due to the availability of NS interfaces at a wide range of slopes, the reflected part will eventually ``find'' a surface with which it is perpendicular to.
Also apparent is the reduction of group velocity; the slope of the bands is greatly reduced. This has an effect on the Majorana coherence length, as described later on in the manuscript.

%%██████████████████████████████████████████████████████████████████████████
%%██ Calculating the topological phase diagram
%%██████████████████████████████████████████████████████████████████████████
\section{Calculating the topological phase diagram.}
In order to create a topological phase diagram we would normally compute the topolical invariant(pfaffian).
For a ribbon this is possible, but computing the invariant of the supercell is too computationally expensive.
Using a generalized eigenvalue problem also proved unsuccesful due to numerical errors, so we resort to simply calculating the gap and noting the gap closings.

Similar to what is described in Pientka et. al [\cite{pientka2017topological}], we see that the ribbon topology has a diamond like structure as a function of phase and Zeeman field.
We also see some additional gap closings which are due to the BDI symmetry class, creating quite an unstable region.
As expected, for the zigzag system, the magnitude of the gap is significantly improved, whilst maintaining the diamond like shape.
As the zigzag shape destroys the additional symmetry, moving it to class D, we see a more clean toplogical region.
\comment{We use a generalized eigenvalue problem to find all phase boundaries at once.}

\comment{We calculate the gap by finding the minimum energy in the spectrum.}

\begin{figure}[!htb]
\includegraphics[width=0.95\columnwidth]{figures/bandstructures}
\caption{Figure of the bandstuctures.
Where the lines are (blue) $\phi=0$, $B=0$ and (red) $\phi=\pi$, $B \ne 0$.
and the subplots (a) $z_y=0$, (b) $z_y=\frac{W}{4}$, and (c) $z_y=\frac{W}{2}$, where $W$ is the junction width.
We observe that once there are no more straight trajectories inside the junction (when $z_y=\frac{W}{2}$) the spectrum becomes most insensitive to the momentum $k_x$.
\label{fig:bandstuctures}}
\end{figure}

\begin{figure}[!htb]
\includegraphics[width=0.95\columnwidth]{figures/phasediagrams}
\caption{Phase diagrams of a straight system (a, b) and zigzag system (c, d), where (a, c) are $E(\mu, B)$ and (b, d) are $E(\phi, B)$.
We use a generalized eigenvalue problem to find all phase boundaries at once and find the minimal energy gap by finding the minimum in the spectrum: $\min{E(k)}$.
\label{fig:phasediagrams}}
\end{figure}

%%██████████████████████████████████████████████████████████████████████████
%%██ Localization lengths and shape effect
%%██████████████████████████████████████████████████████████████████████████
\section{Localization lengths and shape effect}
Using the tight-binding package kwant, we model a finite system to compute the Majorana wave function density for different geometries; ribbon, zigzag, parallel curve and vertically offset sinosoids.
We then superimpose them onto the geometry, as in figure \ref{fig:wavefunctions}.
For the ribbon system, we see that the Majorana coherence length is relatively large, not showing the decolalized nature sought after.
All of the zigzag type geometries however show greatly reduced coherence length.\tabularnewline
This can be explained by two effects: the increase in topological gap, but also the beforementioned reduction of the group velocity.
\comment{We find the Majorana lengths by calculating the slowest decaying mode.}

\begin{figure}[!htb]
\includegraphics[width=0.95\columnwidth]{figures/wavefunctions}
\caption{Wavefunctions for different sizes and geometries.
With (a) a straight system, (b) a zigzag sytem, (c) a system where the normal region is defined by two parallel sine waves, and (d) a system with the Laeven shape.
Inside the figure we indicate the Majorana lenght (or coherence length) $\xi$ and the topological energy gap $E_\textrm{gap}.$
We observe that $\xi$ for (a) is orders of magnitude longer and $E_\textrm{gap}$ orders of magnitude smaller than for (b, c, d), meaning that the details of the geometry do not matter.
\label{fig:wavefunctions}}
\end{figure}

%%██████████████████████████████████████████████████████████████████████████
%%██ Discussion and Conclusions
%%██████████████████████████████████████████████████████████████████████████
\section{Discussion and Conclusions}

\comment{Extensions: we omit several physical effects, disorder, electrostatics, etc.}

\comment{Outlook: how to cope with the complications.}

\comment{Acknowledgments}

We are grateful to "people" for useful discussions.
This work was supported by the Netherlands Organization for Scientific Research (NWO/OCW), as part of the Frontiers of Nanoscience program, the Foundation for Fundamental Research on Matter (FOM), and an ERC Starting Grant STATOPINS 638760.

\bibliographystyle{apsrev4-1}
\bibliography{snakemajoranas}
\end{document}
