\documentclass[english, twocolumn, 10pt, aps, superscriptaddress, floatfix, prb, citeautoscript]{revtex4-1}
\pdfoutput=1
\usepackage[utf8]{inputenc}
\usepackage[T1]{fontenc}
\usepackage{verbatim}
\usepackage{units}
\usepackage{mathtools}
\usepackage{amsmath}
\usepackage{amssymb}
\usepackage{graphicx}
\usepackage{wasysym}
\usepackage{layouts}
\usepackage{siunitx}
\usepackage{bm}
\usepackage{xcolor}
\usepackage[colorlinks, citecolor={blue!50!black}, urlcolor={blue!50!black}, linkcolor={red!50!black}]{hyperref}
\usepackage{bookmark}
\usepackage{tabularx}
\usepackage{microtype}
\usepackage{babel}
\hypersetup{pdfauthor={T. Laeven and friends},pdftitle={Increasing the topological gap by an order of magnitude through the geomerty for Majorana SNS junctions.}}

\setcounter{secnumdepth}{4}
\setcounter{tocdepth}{4}

\DeclareMathOperator{\e}{e}
\DeclareMathOperator{\de}{d\!}
\DeclareMathOperator{\Tr}{Tr}
\DeclareMathOperator{\diag}{diag}
\DeclareMathOperator{\Res}{Res}
\DeclareMathOperator{\sgn}{sgn}
\DeclareMathOperator{\Pf}{Pf}
\DeclareMathOperator{\Det}{Det}
\DeclareMathOperator{\rank}{rank}
\DeclareMathOperator{\im}{Im}
\DeclareMathOperator{\re}{Re}

\renewcommand{\comment}[2]{#2}
% Uncomment the following line for paragraph descriptions to appear in the file.
\renewcommand{\comment}{\paragraph}

\DeclarePairedDelimiter\abs{\lvert}{\rvert}
\DeclarePairedDelimiter\norm{\lVert}{\rVert}

\makeatletter
\let\oldabs\abs
\def\abs{\@ifstar{\oldabs}{\oldabs*}}
\let\oldnorm\norm
\def\norm{\@ifstar{\oldnorm}{\oldnorm*}}
\makeatother

\newcommand{\ev}[1]{\langle#1\rangle}
\newcommand{\bra}[1]{\langle#1|}
\newcommand{\ket}[1]{|#1\rangle}
\newcommand{\bracket}[2]{\langle#1|#2\rangle}

\newcolumntype{L}[1]{>{\raggedright\arraybackslash}p{#1}}
\newcolumntype{C}[1]{>{\centering\arraybackslash}p{#1}}
\newcolumntype{R}[1]{>{\raggedleft\arraybackslash}p{#1}}


\begin{document}


\title{Increasing the topological gap by an order of magnitude through the geomerty for Majorana SNS junctions.}

\author{Tom Laeven}
\affiliation{Kavli Institute of Nanoscience, Delft University of Technology, P.O. Box 4056, 2600 GA Delft, The Netherlands}
\email[Electronic address: ]{tlaeven@hotmail.com}

\author{Bas Nijholt}
\affiliation{Kavli Institute of Nanoscience, Delft University of Technology, P.O. Box 4056, 2600 GA Delft, The Netherlands}
\email[Electronic address: ]{bas@nijho.lt}

\author{Michael Wimmer}
\affiliation{Kavli Institute of Nanoscience, Delft University of Technology, P.O. Box 4056, 2600 GA Delft, The Netherlands}
\affiliation{QuTech, Delft University of Technology, P.O. Box 4056, 2600 GA Delft, The Netherlands}
\email[Electronic address: ]{m.t.wimmer@tudelft.nl}

\author{Anton R. Akhmerov}
\email[Electronic address: ]{snakemajoranas@antonakhmerov.org}
\affiliation{Kavli Institute of Nanoscience, Delft University of Technology, P.O. Box 4056, 2600 GA Delft, The Netherlands}

\date{\today}

\begin{abstract}
We are solving a problem of soft gap in high density SNS Majorana junctions \cite{pientka2017topological}.
We do this by introducing a zigzag geometry.
This works because the maximum trajactory length is cut-off due to the geomerty and therefore the minimal energy gap is given by a simple formula, which we also verify using numerics.
In addition to having a large minimum energy gap, the abscence of long length trajectories means that the localization length is very short as well.
\end{abstract}

\maketitle


\section{Introduction}

\comment{The gap is small because of long trajectories.}

\begin{figure}
% \includegraphics[width=0.95\columnwidth]{figures/SOME_FIGURE}
\caption{Setups. Figure of a straight and zigzag system, including trajectories.
\label{fig:setup}}
\end{figure}


\section{Analytical estimate}

Quasi classical short junction calculation explaination.
Include all, later move stuff to the appendix if necessary.


\section{Numerical bandstructures}

\comment{We consider a 2D SNS junction with a Rashba-BdG Hamiltonian.}

\comment{We discretize the model and simulate it with Kwant.}

\section{Calculating the bandstuctures.}

\section{Calculating the topological phase diagram.}

\comment{We use a generalized eigenvalue problem to find all phase boundaries at once.}

\comment{We calculate the gap by finding the minimum energy in the spectrum.}

\begin{figure}
% \includegraphics[width=0.95\columnwidth]{figures/SOME_FIGURE}
\caption{Figure of the bandstuctures.
Where the lines are (blue) $\phi=0$, $B=0$ and (red) $\phi=\pi$, $B \ne 0$.
and the subplots (a) $z_y=0$, (b) $z_y=\frac{W}{2}$, and (c) $z_y=W$, where $W$ is the junction width.
\label{fig:bandstuctures}}
\end{figure}

\begin{figure}
% \includegraphics[width=0.95\columnwidth]{figures/SOME_FIGURE}
\caption{Phase diagrams of a straight system (a, b) and zigzag system (c, d), where (a, c) are $E(\mu, B)$ and (b, d) are $E(\phi, B)$.
We use a generalized eigenvalue problem to find all phase boundaries at once and find the minimal energy gap by finding the minimum in the spectrum: $\min{E(k)}$.
\label{fig:phasediagrams}}
\end{figure}


\section{Localization lengths and shape effect}

\comment{We find the Majorana lengths by calculating the slowest decaying mode.}

\begin{figure}
% \includegraphics[width=0.95\columnwidth]{figures/SOME_FIGURE}
\caption{Wavefunctions for different sizes and geometries.
With (a) a straight system, (b) a zigzag sytem, (c) a system where the normal region is defined by two parallel sine waves, and (d) a system with the Laeven shape.
Inside the figure we indicate the Majorana lenght (or coherence length) $\xi$ and the topological energy gap $E_\textrm{gap}.$
We observe that $\xi$ for (a) is orders of magnitude longer and $E_\textrm{gap}$ orders of magnitude smaller than for (b, c, d), meaning that the details of the geometry do not matter.
\label{fig:wavefunctions}}
\end{figure}


\section{Discussion and Conclusions}

\comment{Extensions: we omit several physical effects, disorder, electrostatics, etc.}

\comment{Outlook: how to cope with the complications.}

\comment{Acknowledgments}

We are grateful to M. Irfan, L. P. Kouwenhoven, R. M. Lutchyn, C. M. Marcus, T. Ö. Rosdahl, S. Rubbert, D. Sticlet, and D. Varjas for useful discussions.
This work was supported by the Netherlands Organization for Scientific Research (NWO/OCW), as part of the Frontiers of Nanoscience program, the Foundation for Fundamental Research on Matter (FOM), and an ERC Starting Grant STATOPINS 638760.

\bibliographystyle{apsrev4-1}
\bibliography{snakemajoranas}
\end{document}
