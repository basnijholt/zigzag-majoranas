% -*- root: ./report.tex -*-
\chapter{Introduction}
% \comment{We propose that the introduction of a zigzag-shaped modulation of the junction region leads to an increased topological gap, as well as a reduction of the Majorana coherence length.}
% \comment{We hypothesize that the gap improvement is the result of the elimination of long trajectories}

Majorana bound states (MBSs) are a promising candidate to form the basis of a stable platform for topological quantum computing.

\comment{Majoranas are mostly made in hybrid NS structures.}
The current experimental effort is focused on creating pairs of MBSs in hybrid normal-superconductor (NS) structures, where a lengthy strip (or wire) of semiconductor is interfaced with a \cite{lutchyn_majorana_2010,oreg_helical_2010} superconductor.
Within the right parameter regime ($E_z^2>\mu^2+\Delta^2$ for a simple wire), the semiconductor becomes topological and MBSs appear at the ends of the segment.
Recently, a system with a slight modification has been proposed\cite{pientka2017topological}, where instead of having one superconductor, there are two superconductors, creating a superconductors-normal-superconductor (SNS) junction.
This adds an extra knob to adjust, the superconducting phase difference $\phi$, and this should make it easier to tune the system into the topological phase.
In practice, multiple problems arise making even the observation, let alone the manipulation of Majoranas difficult.

\comment{The gap is small because of long trajectories.}
One of the biggest challenges in creating stable Majoranas is the appearance of a soft gap; where the gap in the density of states is greatly reduced for states with the momentum directed along the length of the strip.
From a semiclassical perspective, these momenta correspond to long paths through the semiconductor without interruption by the superconductor.
Additionally, these long trajectories have long flight times $\tau_f$ and equivalenly small Thouless energies $E_{\textrm{Th}}=\hbar / \tau_f$, resulting in a small gap. 

\comment{Currently proposed workarounds are low density or disorder, but both have drawbacks.}
Currently, proposed workarounds are low density or disorder~\cite{haim_double-edge_2018}, but both have drawbacks.

\comment{We show that zigzag geometry solves this problem by eliminating the long trajectories.}
In this paper we show that introducing a zigzag or snake like geometry for the semiconductor eliminates these long trajectories and prevents the appearance of a soft gap while also increasing the topological gap by an order of magnitude.
