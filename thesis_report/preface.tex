% -*- root: ./report.tex -*-
\chapter*{Preface}
\setheader{Preface}

When I started my thesis, I had been out of the quantum physics world for a year and a half, and, to be honest, did not remember much.
This meant that I had many basic questions for which I was quite embarrassed that I had the need to ask.
I was very fortunate to have Bas as a supervisor, who always answered my questions -- however basic -- without making me feel stupid.
It is quite a rare quality in our little bubble: where we all want to show how smart we are.
Apart from that personal note, I am also greatly appreciative to the in-depth support and feedback Bas has provided me: I received a great deal of support in programming, writing, and giving comments on my thesis.
I am also very grateful for setting up the meeting with Anna Keselman, the co-author of the main paper relevant to my thesis, which was very helpful for my understanding.

I would also like to thank Anton and Michael for all the help and discussions, even when I wanted to discuss ideas so vague that the term handwaving sounds too specific.
It was during one of these discussions that I got the suggestion that long trajectories cause a soft gap, which ultimately led to the idea of the zigzag geometry.
I am also very grateful for the opportunities I received to meet and present for researchers in our field outside of the group.

In no particular order, I would like to thank:
Piotr and Sybren for the interesting discussions we had, as well as a fun office atmosphere,
Joe for the many times you helped me with coding, Pablo for answering many questions and KPM, Daniel for helping me with QSymm, Tómas for helping me go through the hell known as scattering matrices, and Srijit, Fokko and Qing for the interesting discussions we had, and kick-ass SEM photographs.

I am also eternally grateful for the creation of Kwant: my favorite quantum physics sandbox.
Using Kwant, I was able to go from idea to implementation to simulation within a matter of hours.
It is truly amazing, and I know for sure that I would have never tried weirdly shaped junctions without it. 

I would also like to thank my girlfriend Irina, for proofreading my thesis, and the detailed comments I received.
I am also very grateful for all the support she gave me, especially in the beginning, when it was quite frustrating, and at the end, when I was stressed.

Finally, I would like to thank my family, who have been extremely supportive of me.
I am especially grateful for them providing a home where I can always come to rest.

\begin{flushright}
{\makeatletter\itshape
    \@author \\
    Delft, February 2019
\makeatother}
\end{flushright}

