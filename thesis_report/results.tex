% -*- root: ./report.tex -*-
\chapter{Results \& Discussion}
	\comment{Introduction of a zigzag opens up the gap, creates a more stable topological phase diagram, and decreases coherence length of Majorana's}
	\section{Increased isolation and stability of the Majorana state}

	\comment{We see that the introduction of a zigzag leads to a reduction of the gap}
		\subsection{Gap/bandstructure as a function of zigzagginess}
			\comment{We discretize the model described in section II and simulate it with Kwant.}
			We implement a tightbinding version of the model described in the Methods section (a two dimensional SNS junction with a Rashba BdG Hamiltonian) using Kwant.

			\begin{figure}[!htb]
			\centering
			\includegraphics[width=0.55\columnwidth]{figures/bandstructures}
			\caption{Figure of the bandstuctures.
			Where the lines are (blue) $\phi=0$, $B=0$ and (red) $\phi=\pi$, $B \ne 0$.
			and the subplots (a) $z_y=0$, (b) $z_y=\frac{W}{4}$, and (c) $z_y=\frac{W}{2}$, where $W$ is the junction width.
			We observe that once there are no more straight trajectories inside the junction (when $z_y=\frac{W}{2}$) the spectrum becomes most insensitive to the momentum $k_x$.
			\label{fig:bandstuctures}}
			\end{figure}

			\comment{We calculate the bandstructure for varying amount of zigzag.}
			In figure~\ref{fig:bandstuctures}, the bandstructure for zigzag systems with varying amplitude is displayed.
			As the zigzag introduces super cell, we see a folded bandstructure.
			For comparisons sake, we take the same cell size for the ribbon structure so the bandstructure is folded in the same way.

			\comment{The bandstructures show that modulation of the geometry increases the gap by an order of magnitude, as well as reduce the group velocity.}
			The introduction of a zigzag has a striking effect: the bands flatten out, and more importantly, the gap size is increased more than an order of magnitude.
			This effect is due to the fact that the modes where the gap is smallest, when the momentum is almost completely focused along the strip ($k_x=k_F$), are cut off by the zigzag geometry.
			Also apparent is the reduction of group velocity; the slope of the bands is greatly reduced.
			This has an effect on the Majorana coherence length, as described later on in the manuscript.

			\comment{The introduction of an SN interface at many angles also increases performance.}
			Additionally in the presence of normal reflection, due to the availability of NS interfaces at a wide range of slopes, the reflected part will eventually ``find'' a surface with which it is perpendicular to. The transparency of the SN interface is thus improved for transversal modes.

		\comment{Introduction of a zigzag leads to localization of the Majorana's}
		\subsection{Majorana decay length}
			\comment{In a zigzag geometry Majoranas are localized within one segment of zigzag.}

			\comment{We find the Majorana lengths by calculating the slowest decaying mode.}
			Using the tight-binding package Kwant~\cite{groth_kwant:_2014}, we model a finite system to compute the Majorana wave function density for different geometries; ribbon, zigzag, parallel curve and vertically offset sinusoids.
			In figure~\ref{fig:wavefunctions} the wavefunctions are displayed, superimposed upon their respective geometry.
			For the ribbon system, we see that the Majorana coherence length is relatively large, not showing the decolalized nature sought after.
			All of the zigzag type geometries however show greatly reduced coherence length.
			This can be explained by two effects: the increase in topological gap, but also the before mentioned reduction of the group velocity.

			\comment{We also confirm that the specific shape does not matter.}
			Also note the similarity between the various shapes, sharp corners or more smoothlike, the effect is the same.

			\begin{figure}[!htb]
			\centering
			\includegraphics[width=0.55\columnwidth]{figures/wavefunctions}
			\caption{Wavefunctions for different sizes and geometries.
			With (a) a straight system, (b) a zigzag system, (c) a system where the normal region is defined by lines parallel to a sinosoid, and (d) a system with vertically offset sinosoids.
			Inside the figure we indicate the Majorana length (or coherence length) $\xi$ and the topological energy gap $E_\textrm{gap}.$
			We observe that $\xi$ for (a) is orders of magnitude longer and $E_\textrm{gap}$ orders of magnitude smaller than for (b, c, d), meaning that the details of the geometry do not matter.
			\label{fig:wavefunctions}}
			\end{figure}

		\comment{Zigzag Majorana's have a cleaner phase diagram due to the reduction in symmetry class}
		\subsection{Phase diagram}
			\comment{We calculate the topological phase diagram using the gap size.}
			We calculate the gap by finding the absolute minimum of the spectrum $E_\textrm{gap}=\min{|E(k)|}$.
			Noting the gap closings and the fact we know the topology for the non-zigzag system, we can infer the topology of the system with zigzag modulation. % Bas: I don't understand what is written here

			\comment{The phase diagram does not change much, except we see a cleaner spectrum as a result of the D class symmetry. }
			Similar to what is described by Pientka et. al~\cite{pientka2017topological}, we see that the ribbon topology has a diamond like structure as a function of phase and Zeeman field.
			We also see some additional gap closings which are due to the BDI symmetry class, creating quite an unstable region.
			As expected, for the zigzag system, the magnitude of the gap is significantly improved, whilst maintaining the diamond like shape.
			As the zigzag shape destroys the additional symmetry, moving it to class D, we see a more clean topological region.

			\begin{figure}[!htb]
			\centering
			\includegraphics[width=0.75\columnwidth]{figures/phasediagrams}
			\caption{Phase diagrams of a straight system (a, b) and zigzag system (c, d), where (a, c) are $E(\mu, B)$ and (b, d) are $E(\phi, B)$.
			We use a generalized eigenvalue problem to find all phase boundaries at once and find the minimal energy gap by finding the minimum in the spectrum: $\min{E(k)}$.
			\label{fig:phasediagrams}}
			\end{figure}

	\section{Finding the mechanism responsible for the increased gap}
		\comment{We use our analytical bandstructure to }
		\subsection{Verification of analytical estimate}
			\begin{figure}[!htb]
			\centering
			\includegraphics[width=0.95\columnwidth]{figures/spectrum_calculation_comparison}
			\caption{Andreev Bound State spectrum of a SNS junction with a tilted in-plane magnetic field, computed using four different methods.}.
			\label{fig:spectrum_calculation_comparison}
			\end{figure}
			
		\subsection{Estimation of the gap through momentum cutoff}

		