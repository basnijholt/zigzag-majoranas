% -*- root: ./report.tex -*-
\chapter{Conclusion}\label{chap:conclusion}
In the SNS system studied, the introduction of a zigzag geometry increases the topological gap by an order of magnitude, while simultaneously reducing the Majorana decay length.
Furthermore, the geometric alteration breaks an additional symmetry, bringing the system from the BDI symmetry class into the D symmetry class, resulting in a cleaner topological phase.

Together, these improvements directly affect the robustness of the Majorana state, potentially bringing us closer to be able to reliably create topological devices.
For instance, the improved topological gap allows for an increased tolerance in device manufacturing, as well as providing a clean background for a hopefully present zero-bias peak.
Also, the protection of the system against thermal fluctuations has an exponential dependence on the gap size, and is thus increased by many orders of magnitude.
Furthermore, the reduced Majorana decay length manifests itself in stronger topological protection, and decreases the minimum required coherence length of the semiconductor.

\comment{With the current quasiclassical analogy, we are not able to explain the improvements of zigzag.}
The current used quasiclassical analogy, even when applied in a very lenient form, is not able to fully explain the increased gap of a zigzag system.
A possible reason for the discrepancy is the fact that the quasiclassical model is not able to capture the availability of a wide range of slopes to scatter against: in a straight system, long trajectories coincide with grazing angles.
Furthermore, numerical simulation points out that the optimal geometry does not always coincide with the minimal trajectory length.
This is possibly explained by tunneling of the wavefunction through the superconductor, but as is the case with the grazing angles, further investigation is necessary.

In any case, the improvements are stable under a wide range of parameters, and the topological gap varies gradually as a function of geometry, indicating the effect is not merely coincidental.
Although we cannot exclude the possibility that some other process is responsible for the gap improvement, we still expect the driving mechanism to be the suppression of long trajectories.
We are currently investigating other strategies to confirm this notion.

\comment{Extensions: we omit several physical effects, disorder, electrostatics, etc.}
In the model used, we omit several physical effects, such as disorder, electrostatics, and the orbital effect.
Although the same restrictions hold for the straight geometry, it would be interesting to see whether inclusion of these physical effects amends the conclusions of this thesis.

\comment{Current fabrication techniques seem compatible with the proposed geometry, and experimental verification should point out whether it holds up to its promise.}
Current fabrication techniques seem compatible with the proposed geometry, and experimental verification should point out whether it holds up to its promise.
We note that slight modifications to the geometry, for example, to ease measurement, should not have large implications on the results.
Currently, experimental groups have already fabricated zigzag devices (e.g. the image on the cover of this thesis), and measurements are underway.

\comment{It would be interesting to study what geometry creates an optimal environment for Majorana modes to exist.}
We have studied sawtooth, snake-like, and offset sinusoidal modulation of the semiconductor region, but perhaps a more exotic shape will be optimal. 
Additionally, although not reported in this thesis, similar results are found for zigzag modulation of classic nanowire systems, where the semiconductor is interfaced with only a single superconductor.
A study optimizing the geometry for supporting Majoranas would be an interesting continuation, and is currently being undertaken within the group.