% -*- root: ./report.tex -*-
\chapter{Conclusion}
In the studied model, the introduction of a zigzag geometry increases the topological gap by over an order of magnitude, as well as offering a substantial reduction of the Majorana length.
Additionally, the breaking of BDI symmetry into D symmetry results in a cleaner topological phase.
Together, these improvements directly affect the robustness of the Majorana state, potentially bringing us closer to reliably create topological devices.

\comment{With the current quasiclassical analogy, we are not able to explain the improvements of zigzag.}
The current used quasiclassical analogy, applied in the most lenient form, is not able to explain the increased gap of a zigzag system.
Furthermore, numerical simulation points out that the optimal geometry does not always coincide with the minimal trajectory length.
It is therefore unsure whether the improvement of the gap can be fully attributed to the elimination of long trajectories, or, if the arising phenomena are due to other mechanisms.
In any case, the improvements are stable under a wide range of parameters, and the topological gap varies gradually as a function of geometry, indicating the effect is not merely coincidental.
Further study should be devoted to isolating the working principle behind the improvements of the zigzag geometry.

\comment{Extensions: we omit several physical effects, disorder, electrostatics, etc.}
In the model used, we omit several physical effects, such as, disorder, electrostatics, the orbital effect, etc.
Apart from experimental verification, further study is required to see whether the conclusions of this model are valid.

\comment{Current fabrication techniques seem compatible with the proposed geometry, and experimental verification should point out whether it holds up to its promise.}
Current fabrication techniques seem compatible with the proposed geometry, and experimental verification should point out whether it holds up to its promise.
We note that slight modifications to the geometry, for example to ease measurement, should not have large implications on the physics.
Currently, experimental groups have already fabricated zigzag devices (e.g. the image on the cover of this thesis), and measurements are underway.

\comment{It would be interesting to study what geometry creates an optimal environment for Majorana modes to exist.}
We have studied sawtooth, snake like and offset sinosoid modulation of the semiconductor region, but perhaps a more exotic shape will be optimal.
A study optimizing the geometry for supporting Majorana's would be an interesting continuation.
