% -*- root: ./report.tex -*-

\chapter{Theoretical Background}\label{chap:theory}
A topological quantum computer uses pairs of Majorana bound states as its unit of computation (qubit).
In order to host Majorana modes in a system, one needs a material which exhibits both superconducting and semiconducting properties.
As this material does not seem to exist on its own, one creates a heterogeneous system.
In simple terms, this consists of a slab of superconductor and a strip of semiconductor.
By interfacing the two, one creates a region where all the necessary ingredients are present to realize Majoranas.

In this chapter, we first give a description of the normal (semiconductor) and superconducting regions.
We then focus on what happens in a system where the two are interfaced.
Following that, we briefly explain what Majoranas are, before introducing the implementation relevant to this thesis.

\section{Normal region}
    Two important ingredients required for realizing Majorana fermions are strong spin-orbit coupling and a strong interaction with the applied magnetic field~\cite{lutchyn_majorana_2018}.
    It is the semiconductor which brings these two physical effects to the mix.
    Note that the Hamiltonian of the normal region is the same for electrons and holes, except for the sign.
    The Hamiltonian terms provided in this section are relevant to the electron states.
    In Figure \ref{fig:ham_simple_terms}, the effect of the various terms in the Hamiltonian on the dispersion relation is shown.

    \begin{figure}[!htb]
    \centering
    \includegraphics[width=0.95\columnwidth]{figures/simple_terms}
    \caption{Energy dispersion of electrons with different physical effects (spin orbit and magnetic field).
    The dashed line in the Figure with both spin-orbit coupling and Zeeman field corresponds to the case when the magnetic field is not aligned with the spin-orbit direction.}
    \label{fig:ham_simple_terms}
    \end{figure}

    \subsection{Kinetic term}
        We rewrite the kinetic term in the Schrodinger equation into its momentum form by noting $- i \partial_x \rightarrow \kx$:
        \begin{equation}
            H_\text{kin} = - \frac{\hbar^2}{2 \meff} \nabla^2 = \frac{\hbar^2}{2 \meff} \left( \kx^2 + \ky^2 \right)
        \end{equation}
        The effective mass is a way of capturing the periodic potential of the crystal lattice into a single term.
        Its value differs per material.

    \subsection{Zeeman interaction}
        The energy splitting of the two electron spin species due to the magnetic field is called the Zeeman interaction.
        It is modeled by:
        \begin{equation}
        H_\text{Z} = g \mu_B \vec{B} \cdot \vec{\sigma}
        \end{equation}

    \subsection{Spin-orbit coupling}
        An electron moving in an electric field will experience a small Zeeman field in its inertial frame as a relativistic effect~\cite{petersen_simple_2000}.
        The resulting interaction, coupling the momentum and spin of an electron, is known as spin-orbit coupling.
        Spin-orbit coupling can be approximated by adding a Rashba spin orbit coupling term in the Hamiltonian:
        \begin{equation}
        H_\text{Rashba} = \alpha (\ky \sigx - \kx \sigy) 
        \end{equation}
        Where $\alpha$ is known as the spin-orbit coupling strength.
        Using this coupling strenght $\alpha$, we can also define the spin-orbit length~\cite{van_weperen_spin-orbit_2015}:
        \begin{equation}
            l_\text{so} = \frac{\hbar^2}{2\meff \alpha}
            \label{eq:l_so}
        \end{equation}
        In a system of size $W$ with Rashba spin-orbit coupling, and $l_\text{so} < W$, the effect of the Rashba term in the Hamiltonian is negligible~\cite{sticlet_robustness_2017}.
\section{Superconducting region}
    The superconductor is responsible for ensuring particle-hole symmetry, as well as providing a `gapped' area in the bandstructure, isolating the zero-energy state.
    Superconductivity is a very complicated phenomenon, so we provide only the background required for understanding the contents of this thesis.

    \subsection{Superconductivity}
        Superconductivity manifests itself as the complete absence of electrical resistance in a material.
        It results from the pairing of electrons close to the Fermi-energy, condensing into so-called Cooper pairs~\cite{nazarov2013advanced}.
        The main parameter defining superconductivity is known as the superconducting order parameter, a complex number.
        The phase of this parameter is a macroscopic potential relative to the choice of gauge.
        The magnitude of the superconducting order parameter, also referred to as the superconducting gap, quantifies the range around the Fermi-energy condensed into Cooper pairs.
        Because all electrons within a distance of the superconducting order parameter to the Fermi-energy are depleted in this way, charge transport only takes place via these quasiparticles.
        Scattering of these Cooper pairs is not allowed, resulting in perfect conduction.

    \subsection{Superconducting Hamiltonian}
        For the purpose of creating Majoranas, we are interested in two main aspects of superconductivity: the energy gap surrounding the Fermi energy (due to the depletion of states), and the electron-hole (also known as particle-hole) symmetry.
        We are thus not interested in the paired up electrons (Cooper pairs), and only wish to model the electronic (unpaired) modes.
        This can be achieved by modeling superconductivity with the Boguliubov de Gennes Hamiltonian, which ensures particle-hole symmetry:
        
        \begin{equation}
        \mathbf{H}_\text{BdG} = \begin{bmatrix} H & \Delta \\ \Delta^\dagger & -H \end{bmatrix}
        \label{eq:bdg}
        \end{equation}

        Note that Eq.\eqref{eq:bdg} describes a block matrix: both $H$ and $\Delta$ can be matrices themselves, where the degrees of freedom correspond to spin, for instance, or discretized positions.
        Using this basis, one is able to describe a heterogeneous system, containing both superconducting and normal regions.
        As an example, we write the Hamiltonian of a heterogeneous 1D system, consisting of a superconductor extending from $x = -\infty$ to $0$, and a normal system from  $x = 0$ to $\infty$:

        \begin{equation}
            \mathbf{H} =  \Theta \left(-x\right) \begin{bmatrix} H_\text{SC, kin} & \Delta \\ \Delta^\dagger & -H_\text{SC, kin} \end{bmatrix} + \Theta \left(x\right)  \left( \tau_z \otimes H_\text{normal} \right) 
            \end{equation}
        where, 
        \begin{equation}
            \Theta(x) = \begin{cases} 1 & x \geq 0 \\ 0 & x < 0 \end{cases}
        \end{equation}


    \subsection{Proximity effect}
        In order to create Majoranas, one needs a material with properties unique to both superconductors and semiconductors.
        However, if a superconductor is interfaced with a normal conductor, some of the properties belonging to the superconductor transfer to the normal material.
        This effective transfer of properties to the normal material is known as the proximity effect.
        The characteristic distance governing the decay of the superconducting behavior is given by the superconducting coherence length:
        
        \begin{equation}
            \xi = \hbar \frac{v_\text{F}}{\Delta}
            \label{eq:theory_sc_coherence_length}
        \end{equation}

\section{Normal-superconducting junctions}
    As mentioned, interesting physics occur when a normal conductor is interfaced with a superconductor.
    A single interface between a normal conductor and a superconductor is known as an NS junction.
    When one creates a sandwich, superconductor - normal - superconducting, we speak of an SNS junction.
    Even when topological phases are not considered, which are only present if the normal regions meet certain conditions, both have interesting physics which will be discussed first.


    \subsection{Andreev scattering}
        At the interface between a normal and superconducting region, electrons with energies below the superconducting gap undergo Andreev scattering.
        The quasiclassical microscopic explanation for this is as follows.
        An electron in the normal region with energy below the superconducting gap, incident on the normal-superconductor (NS) interface, is retro-reflected as a hole.
        This phenomenon is known as Andreev scattering and occurs due to the coupling of electrons in a superconductor.
        Although a single electron initially enters the superconductor, it draws another electron (with opposite momentum) into the superconductor, thus forming a Cooper pair.
        The drawn in electron leaves behind a hole in the normal conductor, which essentially backtracks the motion of the initial electron, hence the term retro-reflection.
        The same effect, but opposite in particle type, holds for a hole entering the superconductor.

    \subsection{Andreev bound states}
        If the normal region is finite in the direction normal to the NS interface, Andreev bound states form.
        This is true for both SNS and SN junctions.
        Quasiclassically, one can imagine electrons (and holes) in the normal region, with energies below the gap, bouncing back and forth the system.
        The corresponding eigenstates have energies below the gap and are thus bound to the normal region.

    \subsection{Josephson junctions and supercurrent}
        An important device in the field of condensed matter physics is the Josephson junction.
        A Josephson junction consists of a normal region sandwiched by two superconductors (SNS junction).
        The resulting Andreev bound states exhibit interesting physics, making the Josephson junction an important device.

        Although the superconducting phase is relative to the choice of gauge, the difference of phase between two superconductors is a physical quantity.
        A difference in superconducting phase in a Josephson junction results in a flow of current between the two superconductors: a supercurrent.
        The relationship between the superconducting phase and current is known as the current-phase relationship, and is in the simplest case a sinusoid.
        This relationship can be exploited to control the phase difference: it is a matter of inducing a current.

        An important measure in a Josephson junction is the Thouless energy, given by Equation~\eqref{eq:theory_E_th}:
        \begin{equation}
            E_\text{Th} = \frac{\hbar v_F}{W}
            \label{eq:theory_E_th}
        \end{equation}
        Where $v_F$ is the Fermi velocity, and $W$ is the width (distance between superconductors) of the system.

        As an example of its relevance, consider a SNS junction with a magnetic field such that $E_Z \approx E_\text{Th}$.
        An electron traversing the system from superconductor to superconductor will, under influence of the magnetic field, have its spin flipped.

    
\section{Topology, Symmetry, and Majorana bound states}
    In condensed matter physics, topology refers to the study of properties of a system, which are guaranteed by particular symmetries of the Hamiltonian.
    These symmetries are present for a particular range of parameters such that their variation does not destroy these \emph{topologically protected} properties.
    
    Topology is relevant to the context of the system under study because the emergence of separated Majoranas is guaranteed within a continuous range of parameters.
    The presence of Majoranas is therefore topologically protected, as they will be unperturbed by variations of parameters, within a certain range.
    For an infinite system, the system is said to be in the topological phase if Majoranas emerge once the system is cut to finite length.
    The system is considered to be in its trivial phase if the converse is true.
    The allocation of a system into one of its phases is quantitatively captured by its topological invariant.
    The relevant topological invariant is dependent on the symmetry class the system belongs to.
    In the case considered in this thesis, the systems discussed belong in either the symmetry class BDI or D.
    Any system which is in the BDI class is also contained within the D class.
    The topological invariant of a class D system is binary: $-1$, corresponding to the topological phase, and $1$, corresponding to the trivial phase.
    The invariant of a BDI system can be any integer number, which also means there are more than two topological phases.
    More importantly, the transition of the system from one topological phase to the other means the system must undergo a gap closing.


    \subsection{Majorana bound states}
        We will give a very pragmatic description of Majorana bound states, insofar their relevance to this work.

        The Majorana state is a superposition of electrons and holes, and is its own anti-particle: exchanging electrons and holes yields the same state.
        The Majorana state carries no spin, charge, or energy, giving almost no degrees of freedom for it to interact with the environment.
        The Majorana state constitutes a single fermion but is comprised of two Majoranas.
        In a finite sized topological system, these two Majoranas are physically separated, appearing at the edges.

        The lack of interaction, as well as the physical separation of the state, makes it very robust against perturbations, but also difficult to detect.
        Fortunately, the state's presence itself is measurable: the availability of a state contributes to the conduction at that energy.
        When measuring the conduction through a system in which Majoranas are present, one should thus encounter a rise in the conductance at zero energy: the zero bias peak.

    \subsection{Topological gap}
        The topological gap is defined as the energy gap between the Majorana state, conveniently located at zero energy, and the first available state.
        Its magnitude governs the protection of the system against perturbations causing the system to be lifted out of its ground state: the Majorana state.

    \subsection{Decay/Coherence length}
        The decay or coherence length of a Majorana pair is an important measure describing the separation of the two half-particles.
        A decay length larger equal in size to the system dimensions will lead to overlap of the Majoranas, diminishing the desired robustness of the state.
        An overlap of the Majoranas will result in the system having non-zero energy, its magnitude depending on the amount of overlap.
        A crude estimation of the size of an individual Majorana is:
        \begin{equation}
            \xi = \hbar \frac{v_\text{F}}{\Delta_\text{topological}}
            \label{eq:majorana_coherence_length}
        \end{equation}

        When the system is cut to a finite length, any overlap of the two Majoranas will lead to non-zero energy.
        This decreases the robustness of the topological state and it is thus desirable that the Majorana decay length is significantly smaller than the system size.

\section{ Two-dimensional platform for Majoranas}

    \begin{figure}[!htb]
    \centering
    \includegraphics[width=0.5\columnwidth]{figures/pientka_system}
    \caption{System overview of the paper by Pientka et. al.~\cite{pientka_topological_2017}
    A magnetic field is directed along the length of the strip, and the separation between the two superconductors is $W$.
    A superconducting phase difference lifts the mirror-symmetry otherwise present in the system, allowing for the system to enter the topological phase.
    In a finite system of size $L$, Majoranas (represented by $\gamma$) appear at the ends of the strip.
    Note that in this system, the two superconducting slabs (blue) are deposited onto the semiconductor, proximitizing the 2DEG below.
    By doing so, the underlying areas become the effective superconductors, and the entire system is defined within the semiconductor crystal, ensuring virtually perfect NS interfaces.\\
    Image taken from \cite{pientka_topological_2017} licensed under CC BY 4.0.
    }
    \label{fig:pientka_system}
    \end{figure}

    The earliest devices used to create a separate pair of Majoranas were nanowires deposited onto a slab of superconducting material.
    A relatively new proposal is that of creating Majoranas in a planar Josephson junction~\cite{pientka_topological_2017}.
    An advantage of this system is that the phase difference between the superconductors can be used as a knob to switch the system from the topological state to the trivial state. 
    In addition, the presence of the second superconductor increases the topological gap of the system due to the proximity effect.

    \subsection{Phase diagram}
        As mentioned, the superconducting phase difference plays a critical role in the physics of the device.
        The topological phase diagram depicted in Figure~\ref{fig:pientka_phase_diagram}, displays the topological phase of the system as a function of magnetic field and superconducting phase.
        Without normal reflection (perfect Andreev reflection), the diagram has a diamond-like shape.
        The system is always topological when the superconducting phase is $\pi$, or when the magnetic field is tuned to $(2N-1)$ times the Thouless energy(Eq. \ref{eq:theory_E_th}).
        Given the relationship between superconducting current and phase, control of the current allows for switching the system between the trivial phase and the topological phase.
        
        \begin{figure}[!htb]
        \centering
        \includegraphics[width=0.5\columnwidth]{figures/pientka_phase_diagram}
        \caption{Topological phase diagram as a function of superconducting phase and magnetic field.
        The diagram has a diamond shape, repeating along the magnetic field axis with a period equal to $2 E_\text{Th}$.\\
        Image taken from \cite{pientka_topological_2017} licensed under CC BY 4.0.
        }
        \label{fig:pientka_phase_diagram}
        \end{figure}
            
    \subsection{Bandstructure of a typical system}
        In Figure \ref{fig:pientka_bandstructure}, the bandstructure is displayed for the SNS system.
        Note that the gap attains the lowest value at the Rashba/Zeeman split Fermi-surface, where the momentum in the $x$-direction equals $k_{F,1}$ or $k_{F,2}$.
        It is at these momenta that the wavefunctions are completely parallel to the translationally invariant direction, causing weak coupling with the superconductors.

        \begin{figure}[!htb]
        \centering
        \includegraphics[width=0.5\columnwidth]{figures/pientka_bandstructure}
        \caption{Bandstructure of a straight SNS system.
        Note that the gap is minimized when $k_x=k_{F,1}$ and $k_x=k_{F,2}$, which correspond to the Rashba/Zeeman split Fermi momenta.\\
        Image taken from \cite{pientka_topological_2017} licensed under CC BY 4.0.
        }
        \label{fig:pientka_bandstructure}
        \end{figure}