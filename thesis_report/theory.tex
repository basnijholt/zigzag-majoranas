% UNCOMMENT SO YOU CAN LINK PAPERS EASILY WITH SUBLIME TEXT
% \bibliography{report}


\chapter{Theory}


\section{General Physics}

    \subsection{Zeeman interaction}
    	The energy splitting of the two electron spin species due to magnetic field is called the Zeeman interaction.
    	It is modeled by:
    	\begin{equation}
    	H_\text{Z} = g \mu_B \vec{B} \cdot \vec{\sigma}
    	\end{equation}

    \subsection{Spin-orbit coupling}
	    An electron mvoing in an electric field will experience a small Zeeman field in its inertial frame as a relativistic effect~\cite{petersen_simple_2000}.
	    The resulting interaction is known as spin orbit coupling.
	    Spin-orbit coupling can be approximated with adding a Rashba spin orbit coupling term in the Hamiltonian:
	    \begin{equation}
	    H_\text{Rashba} = \alpha (\ky \sigx - \kx \sigy) 
	    \end{equation}

    
    \subsection{Superconductivity}
		Superconductivity manifests itself as the complete lack of electrical resistance of a material.
		It is the result of the pairing of electrons close to the Fermi-energy, condensing into so-called Cooper pairs.
		Because all electrons within a distance of the superconducting order parameter of the Fermi-energy are depleted in this way, charge transport only takes place via these quasiparticles.
		Scattering of these Cooper pairs is not allowed, resulting in perfect conduction.

		For the purpose of creating Majorana's, we are interested in two main aspects of superconductivity: the energy gap surrounding the Fermi energy (due to the depletion of states), and the electron-hole (/particle-hole) symmetry.
		We are thus not interested in the paired up electrons (Cooper pairs), and only wish to model the electronic (unpaired) modes.
		This can be achieved by modeling superconductivity with the Boguliubov de Gennes Hamiltonian:
		
		\begin{equation}
		\mathbf{H}_\text{BdG} = \begin{bmatrix} H & \Delta \\ \Delta^\dagger & H \end{bmatrix}
		\end{equation}

		The block matrix restricts the structure of the system in the electron-hole basis, and ensures particle-hole symmetry.
		For $\Delta \neq 0$ one indeed induces a identically sized gap in the spectrum.

		\subsubsection{Andreev scattering}
			At the interface between a normal and superconducting region, electrons with energies below the superconducting gap undergo Andreev scattering.
			The quasiclassical microscopic explanation for this is as follows.
			An electron in the normal region with energy below the superconducting gap, incident on the normal-superconductor (NS) interface, is retro-reflected as a hole.
			This phenomenon is known as Andreev scattering, and occurs due to the coupling of electrons in a superconductor.
			Although a single electron initially enters the superconductor, it draws another electron (with opposite momentum) into the superconductor: forming a Cooper pair.
			The drawn in electron leaves behind a hole in the normal conductor, which essentially backtracks the motion of the initial electron, hence the term retro-reflection.
			The same effect, but opposite in particle type, holds for a hole entering the superconductor.

		\subsubsection{Josephson junctions and supercurrent}
			

\section{Topology and Majorana bound states}
	\subsection{Topology in solid state physics}
	The Wikipedia definition of mathematical topology is as follows:
	
	\quote{Topology can be formally defined as "the study of qualitative properties of certain objects (called topological spaces) that are invariant under a certain kind of transformation}\cite{noauthor_topology_2018}

		\subsubsection{Symmetries}

	\subsection{General properties}
	\subsection{Decay length}
	\subsection{Topological gap}


\section{Two dimensional platform for Majoranas}
	\subsection{Phase diagram}
	\subsection{Current phase relationship}
	\subsection{Thouless energy}
	\subsection{Transparency at NS interface}

\section{Scattering matrix formalism}

