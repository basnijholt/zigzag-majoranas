% -*- root: ./report.tex -*-

\chapter{Theory}

\section{General Physics}
	We give a brief description of the interactions and physics present in the system under study.

    \subsection{Zeeman interaction}
    	The energy splitting of the two electron spin species due to magnetic field is called the Zeeman interaction.
    	It is modeled by:
    	\begin{equation}
    	H_\text{Z} = g \mu_B \vec{B} \cdot \vec{\sigma}
    	\end{equation}

    \subsection{Spin-orbit coupling}
	    An electron mvoing in an electric field will experience a small Zeeman field in its inertial frame as a relativistic effect~\cite{petersen_simple_2000}.
	    The resulting interaction is known as spin orbit coupling.
	    Spin-orbit coupling can be approximated with adding a Rashba spin orbit coupling term in the Hamiltonian:
	    \begin{equation}
	    H_\text{Rashba} = \alpha (\ky \sigx - \kx \sigy) 
	    \end{equation}

    
    \subsection{Superconductivity}
		Superconductivity manifests itself as the complete lack of electrical resistance of a material.
		It is the result of the pairing of electrons close to the Fermi-energy, condensing into so-called Cooper pairs.
		The main parameter defining superconductivity is known as the superconducting order parameter, a complex number.
		The phase of this parameter is a macroscopic potential relative to ones choice of gauge.
		The magnitude of the superconducting order parameter, also referred to as the superconducting gap, quantifies the range around the Fermi-energy condensed into Cooper pairs.
		Because all electrons within a distance of the superconducting order parameter of the Fermi-energy are depleted in this way, charge transport only takes place via these quasiparticles.
		Scattering of these Cooper pairs is not allowed, resulting in perfect conduction.

		For the purpose of creating Majorana's, we are interested in two main aspects of superconductivity: the energy gap surrounding the Fermi energy (due to the depletion of states), and the electron-hole (/particle-hole) symmetry.
		We are thus not interested in the paired up electrons (Cooper pairs), and only wish to model the electronic (unpaired) modes.
		This can be achieved by modeling superconductivity with the Boguliubov de Gennes Hamiltonian:
		
		\begin{equation}
		\mathbf{H}_\text{BdG} = \begin{bmatrix} H & \Delta \\ \Delta^\dagger & H \end{bmatrix}
		\end{equation}

		The block matrix restricts the structure of the system in the electron-hole basis, and ensures particle-hole symmetry.
		For $\Delta \neq 0$ one indeed induces a identically sized gap in the spectrum.

		\subsubsection{Andreev scattering and bound states}
			At the interface between a normal and superconducting region, electrons with energies below the superconducting gap undergo Andreev scattering.
			The quasiclassical microscopic explanation for this is as follows.
			An electron in the normal region with energy below the superconducting gap, incident on the normal-superconductor (NS) interface, is retro-reflected as a hole.
			This phenomenon is known as Andreev scattering, and occurs due to the coupling of electrons in a superconductor.
			Although a single electron initially enters the superconductor, it draws another electron (with opposite momentum) into the superconductor: forming a Cooper pair.
			The drawn in electron leaves behind a hole in the normal conductor, which essentially backtracks the motion of the initial electron, hence the term retro-reflection.
			The same effect, but opposite in particle type, holds for a hole entering the superconductor.


		\subsubsection{Josephson junctions, Andreev bound states, and supercurrent}
			An important device in the field of condensed matter physics is the Josephson junction.
			A Josephson junction consists of a normal region sandwhiched in between two superconductors (SNS junction).
			Electrons within the normal region, with energies below the superconducting gap, experience Andreev scattering from both superconductors.
			The eigen modes of the sandwiched system, with energies below the gap, are known as Andreev bound states.
			It is these Andreev bound states which exhibit interesting physics, and make the Josephson junction such an important device.

			Although superconducting phase is relative to the choice of gauge, the difference of phase between two superconductors is in fact a physical quantity.
			A difference in superconducting phase results in a flow of current from one superconductor to the other, known as a supercurrent.
			The relationship between superconducting phase and current is known as the current-phase relationship, and is in the most simple case a sinosoid.
			This relationship is exploited by experimentalist to control the phase difference between the two superconductors.
			The amplitude of the current variation due to variation of phase is known as the critical current, and is the relatively trivial physical quantity to measure.
			
			Josephson junctions are relevant to this thesis as the primary device studied is a Josephson junction itself.

		\subsubsection{Thouless energy}


\section{Topology and Majorana bound states}

	The Wikipedia definition of mathematical topology is as follows:
	
	\begin{quote}Topology can be formally defined as "the study of qualitative properties of certain objects (called topological spaces) that are invariant under a certain kind of transformation
	\cite{noauthor_topology_2018}
	\end{quote}

	Similarly, in condensed matter physics, topology refers to the study of properties of a system, which do not change within a particular range of parameters.
	Topology is relevant to the context of the system under study, because the emergence of seperated Majorana's is guaranteed within a continuous range of parameters.
	The precense of Majorana's is therefore \emph{topologically protected}, as they will be unpurturbed by variations of parameters, within a certain range.
	For an infinite system, the system is said to be topological if Majorana's would emerge if the system is cut to finite length.
	The system is trivial if the converse is true.

	\subsection{Majorana bound states}
		We will give a very pragmatic description of Majorana bound states, insofar their relevance to this work.

		The Majorana state is a superposition of electrons and holes, and is its own anti-particle: exchanging electrons and holes yields the same state.
		The Majorana state carries no spin, charge, or energy, giving almost no degrees of freedom for it to interact with the environment.
		The Majorana state constitutes a single fermion, but is comprised of two Majorana's.
		In a finite, topological system, these two Majorana's are physically seperated, appearing at the edges.

		The lack of interaction, as well as the physical seperation of the state, makes it very robust against perturbations, but also difficult to detect.
		Luckily, the state's presence itself is measurable: the availability of a state contributes to the conduction at that energy.
		When measuring the conduction through a system with Majorana's present, one should thus encounter a rise in the conductance at zero energy: the zero bias peak.

	\subsection{Decay/Coherence length}
		The decay or coherence length of a Majorana pair is an important measure signifying the seperation of the two half-particles.
		A decay length larger or of equal size to the system dimensions will lead to overlap of the Majorana's, diminishing the desired robustness of the state.
		An overlap of the Majorana's will in fact lead to the system having a non-zero energy, its magnitude depending on the amount of overlap.
		A crude estimation of the size of an individual Majorana is:
		\begin{equation}
			\xi = \hbar \frac{v_\text{F}}{\Delta}
		\end{equation}

	\subsection{Topological gap}
		The topological gap is defined as the energy gap between the Majorana state -- conveniently located at zero energy -- and the first available state.
		It's magnitude governs the protection of the system against perturbations causing the system to be lifted out of its ground state: the Majorana state.

\section{Two dimensional platform for Majoranas}

	\begin{figure}[!htb]
	\centering
	\includegraphics[width=0.5\columnwidth]{figures/pientka_system}
	\caption{System overview. Source\cite{pientka_topological_2017}}
	\label{fig:pientka_system}
	\end{figure}

	The earliest devices used to create a seperated pair of Majorana's were nanowires deposited onto a slab of superconducting material.
	A relatively new proposal is that of creating Majorana's in a planar Josephson junction~\cite{pientka_topological_2017}.
	An advantage of this system is that the phase difference between the superconductors can be used as a knob to switch the system from the topological state to the trivial state. 
	In addition, the presence of the second superconductor increases the topological gap of the system due to the proximity effect.

			

	\subsection{Phase diagram}
	As mentioned, the superconducting phase difference plays a critical role in the physics of the device.
	The topological phase diagram, as a function of magnetic field and superconducting phase, is depicted in Fig. \ref{fig:pientka_phase_diagram}.
	Without normal reflection (perfect Andreev reflection), the diagram has a diamond-like shape.
	The system is always topological when the superconducting phase is $pi$, or when the magnetic field is tuned to $(2N-1)$ times the Thouless energy (With exception of the corners of the diamonds).
	
	\begin{figure}[!htb]
	\centering
	\includegraphics[width=0.5\columnwidth]{figures/pientka_phase_diagram}
	\caption{Phase vs. Magnetic field. Source\cite{pientka_topological_2017}}
	\label{fig:pientka_phase_diagram}
	\end{figure}
	

	\subsection{Current phase relationship}
	
	
	\subsection{Transparency at NS interface}
