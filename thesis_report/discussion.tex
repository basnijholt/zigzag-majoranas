\chapter{Discussion \& Results}
	
	\comment{We see that the introduction of a zigzag leads to a reduction of the gap}
	\section{Gap/bandstructure as a function of zigzagginess}
		\comment{We discretize the model described in section II and simulate it with Kwant.}
		We implement a tightbinding version of the model described in section II (a two dimensional SNS junction with a Rashba BdG Hamiltonian) using Kwant.

		\begin{figure}[!htb]
		\includegraphics[width=0.95\columnwidth]{figures/bandstructures}
		\caption{Figure of the bandstuctures.
		Where the lines are (blue) $\phi=0$, $B=0$ and (red) $\phi=\pi$, $B \ne 0$.
		and the subplots (a) $z_y=0$, (b) $z_y=\frac{W}{4}$, and (c) $z_y=\frac{W}{2}$, where $W$ is the junction width.
		We observe that once there are no more straight trajectories inside the junction (when $z_y=\frac{W}{2}$) the spectrum becomes most insensitive to the momentum $k_x$.
		\label{fig:bandstuctures}}
		\end{figure}

		\comment{We calculate the bandstructure for varying amount of zigzag.}
		In figure~\ref{fig:bandstuctures}, the bandstructure for zigzag systems with varying amplitude is displayed.
		As the zigzag introduces super cell, we see a folded bandstructure.
		For comparisons sake, we take the same cell size for the ribbon structure so the bandstructure is folded in the same way.

		\comment{The bandstructures show that modulation of the geometry increases the gap by an order of magnitude, as well as reduce the group velocity.}
		The introduction of a zigzag has a striking effect: the bands flatten out, and more importantly, the gap size is increased more than an order of magnitude.
		This effect is due to the fact that the modes where the gap is smallest, when the momentum is almost completely focused along the strip ($k_x=k_F$), are cut off by the zigzag geometry.
		Also apparent is the reduction of group velocity; the slope of the bands is greatly reduced.
		This has an effect on the Majorana coherence length, as described later on in the manuscript.

		\comment{The introduction of an SN interface at many angles also increases performance.}
		Additionally in the presence of normal reflection, due to the availability of NS interfaces at a wide range of slopes, the reflected part will eventually ``find'' a surface with which it is perpendicular to. The transparency of the SN interface is thus improved for transversal modes.


	\section{Majorana decay length}
	\section{Phase diagram for zigzaggy system}
	\section{Analytical estimate}
